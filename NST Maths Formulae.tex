\documentclass[fleqn]{article}
\usepackage[a4paper, left=25.4mm, top=25.4mm, right=25.4mm, bottom=25.4mm]{geometry}
\usepackage[shortlabels]{enumitem}
\usepackage{amsmath}
\usepackage{amssymb}
\usepackage{authoraftertitle}
\usepackage{blindtext}
\usepackage{bm}
\usepackage{color}
\usepackage{graphicx}
\usepackage{newpxtext}
%\usepackage{newpxmath}
\usepackage{textcomp}
\newcommand\tab[1][1cm]{\hspace*{#1}}

\author{Victor Zhao\\xz398@cam.ac.uk}

\begin{document}
\centering
\section*{NST Part IA Mathematics}
\MyAuthor

\begin{enumerate}
    \item Parallel and perpendicular components of a vector:\\
        $\mathbf{a}_\parallel=(\mathbf{a}\cdot\hat{\mathbf{n}})\hat{\mathbf{n}}$\\
        $\mathbf{a}_\perp=\mathbf{a}-(\mathbf{a}\cdot\hat{\mathbf{n}})\hat{\mathbf{n}}$
    \item Vector triple product:\\
        $[\mathbf{a},\mathbf{b},\mathbf{c}]=\mathbf{a}\cdot(\mathbf{b}\times\mathbf{c})$\\
        $[\mathbf{a},\mathbf{b},\mathbf{c}]=[\mathbf{b},\mathbf{c},\mathbf{a}]=[\mathbf{c},\mathbf{a},\mathbf{b}]$\\
        $[\mathbf{a},\mathbf{b},\mathbf{c}]=-[\mathbf{a},\mathbf{c},\mathbf{b}]$
    \item $\mathbf{a}\times(\mathbf{b}\times\mathbf{c})=(\mathbf{a}\cdot\mathbf{c})\mathbf{b}-(\mathbf{a}\cdot\mathbf{b})\mathbf{c}$
    \item Plane equation:\\
        $\mathbf{r}\cdot\hat{\mathbf{n}}=\mathbf{a}\cdot\hat{\mathbf{n}}=d$\\
        $|$$d$$|$ is perpendicular distance of plane from origin for unit normal $\hat{\mathbf{n}}$
    \item Polar coordinates:\\
        $\hat{\mathbf{r}}=\cos\phi\mathbf{i}+\sin\phi\mathbf{j}$\\
        $\hat{\bm{\phi}}=-\sin\phi\mathbf{i}+\cos\phi\mathbf{j}$\\
        $dS=rdrd\phi$
    \item Cylindrical coordinates:\\
        $x=r\cos\phi$\\
        $y=r\sin\phi$\\
        $z=z$\\
        $dV=rdrd\phi dz$
    \item Spherical coordinates:\\
        $x=r\sin\theta\cos\phi$\\
        $y=r\sin\theta\sin\phi$\\
        $z=r\cos\theta$\smallbreak
        $\hat{\mathbf{r}}=\sin\theta\cos\phi\mathbf{i}+\sin\theta\sin\phi\mathbf{j}+\cos\theta\mathbf{k}$\\
        $\hat{\bm{\theta}}=\cos\theta\cos\phi\mathbf{i}+\cos\theta\sin\phi\mathbf{j}-\sin\theta\mathbf{k}$\\
        $\hat{\bm{\phi}}=-\sin\phi\mathbf{i}+\cos\phi\mathbf{j}$\smallbreak
        $dV=r^2\sin\theta dr d\theta d\phi$\\
        $dS=r^2\sin\theta d\theta d\phi$
    \item Leibnitz's formula:\smallbreak
        $\dfrac{d^n(fg)}{dx^n}=\displaystyle\sum_{i=0}^{n}\binom{n}{i}f^{(n-i)}g^{(i)}$
    \item Limits: \smallbreak
        $\displaystyle\lim_{x\to a}f(x)=K$ means that
        $\forall\epsilon>0.\exists\delta>0.\;(0<$ $|$$x-a$$|$ $<\delta)\implies($ $|$$f(x)-K$$|$ $<\epsilon)$\smallbreak
        $\displaystyle\lim_{x\to a+}f(x)=K$ means that
        $\forall\epsilon>0.\exists\delta>0.\;(0<x-a<\delta)\implies($ $|$$f(x)-K$$|$ $<\epsilon)$\smallbreak
        $\displaystyle\lim_{x\to a-}f(x)=K$ means that
        $\forall\epsilon>0.\exists\delta>0.\;(0<a-x<\delta)\implies($ $|$$f(x)-K$$|$ $<\epsilon)$\smallbreak
        $\displaystyle\lim_{x\to\infty}f(x)=K$ means that
        $\forall\epsilon>0.\exists X>0.\;(x>X)\implies($ $|$$f(x)-K$$|$ $<\epsilon)$
    \item Continuity: $f(x)$ is continuous at $a$ if:
        \begin{itemize}
            \item $f(a)$ exists;
            \item $\displaystyle\lim_{x\to a}f(x)$ exists and equals $f(a)$.
        \end{itemize}
    \newpage
    \item Taylor's series:\smallbreak
        $e^x=1+x+\dfrac{x^2}{2!}+\cdots+\dfrac{x^n}{n!}+\cdots$\smallbreak
        $\ln(1+x)=x-\dfrac{x^2}{2}+\dfrac{x^3}{3}-\cdots+(-1)^{n+1}\dfrac{x^n}{n}+\cdots$\smallbreak
        $\sin x=x-\dfrac{x^3}{3!}+\dfrac{x^5}{5!}-\cdots+(-1)^n\dfrac{x^{2n+1}}{(2n+1)!}+\cdots$\smallbreak
        $\cos x=1-\dfrac{x^2}{2!}+\dfrac{x^4}{4!}-\cdots+(-1)^n\dfrac{x^{2n}}{(2n)!}+\cdots$\smallbreak
        $\tan^{-1}x=x-\dfrac{x^3}{3}+\dfrac{x^5}{5}-\cdots+(-1)^n\dfrac{x^{2n+1}}{2n+1}+\cdots$\smallbreak
        $\sinh x=x+\dfrac{x^3}{3!}+\dfrac{x^5}{5!}+\cdots+\dfrac{x^{2n+1}}{(2n+1)!}+\cdots$\smallbreak
        $\cosh x=1+\dfrac{x^2}{2!}+\dfrac{x^4}{4!}+\cdots+\dfrac{x^{2n}}{(2n)!}+\cdots$\smallbreak
        $\tanh^{-1}x=x+\dfrac{x^3}{3}+\dfrac{x^5}{5}+\cdots+\dfrac{x^{2n+1}}{2n+1}+\cdots$
    \item Hyperbolic functions:\smallbreak
        $\cosh^2 x-\sinh^2 x=1$\smallbreak
        $\cosh2x=\cosh^2 x+\sinh^2 x$\smallbreak
        $\cosh^{-1}x=\ln(x+\sqrt{x^2-1})$\smallbreak
        $\sinh^{-1}x=\ln(x+\sqrt{x^2+1})$\smallbreak
        $\tanh^{-1}x=\dfrac{1}{2}\ln\left(\dfrac{1+x}{1-x}\right)$
    \item Differentiation of integrals wrt parameters:\smallbreak
        $\dfrac{d}{dt}\displaystyle\int_{a(t)}^{b(t)}f(x,t)dx=\displaystyle\int_{a(t)}^{b(t)}\dfrac{\partial f}{\partial t}dx+\dfrac{db}{dt}f(b,t)-\dfrac{da}{dt}f(a,t)$
    \item Schwarz's inequality:\smallbreak
        $\left(\displaystyle\int_a^b f(x)g(x)dx\right)^2\leq\left(\displaystyle\int_a^b f^2(x)dx\right)\left(\displaystyle\int_a^b g^2(x)dx\right)$
    \item Gaussian integral:\smallbreak
        $\displaystyle\int_{-\infty}^{+\infty}e^{-x^2}dx=\sqrt{\pi}$
    \item Bayes' theorem:\\
        $P(A|B)=\dfrac{P(B|A)P(A)}{P(B)}$
    \item Poisson distribution:\\
        $P(X=r)=e^{-\lambda}\dfrac{\lambda^r}{r!}$\\
        mean = variance = $\lambda$
    \item Lifetime distribution:\\
        $f(t)=\lambda e^{-\lambda t}$\\
        mean = $\frac{1}{\lambda}$, variance = $\frac{1}{\lambda^2}$\\
    \item Gaussian (normal) distribution:\\
        $f(x)=\dfrac{1}{\sqrt{2\pi\sigma^2}}e^{-\frac{(x-\mu)^2}{2\sigma^2}}$\\
        mean = $\mu$, variance = $\sigma^2$
    \item Linear 1st-order ODE:\smallbreak
        $\dfrac{dy}{dx}+p(x)y=f(x)$:\smallbreak
        $y=\dfrac{1}{\mu(x)}\displaystyle\int\mu(x)f(x)dx$, where $\mu(x)=e^{\int p(x)dx}$
    \item Linear 2nd-order ODE (constant coefficients):\smallbreak
        $\dfrac{d^2y}{dx^2}+a\dfrac{dy}{dx}+by=f(x)$:\smallbreak
        Complementary function: consider the auxiliary equation $\lambda^2+a\lambda+b=0$:
        \begin{itemize}[noitemsep, topsep=0pt]
            \item Case of real, distinct roots $\lambda_1$, $\lambda_2$: $y_c=Ae^{\lambda_1x}+Be^{\lambda_2x}$
            \item Case of equal roots $\alpha$: $y_c=(A+Bx)e^{\alpha x}$
            \item Case of complex roots $\alpha\pm\beta i$: $y_c=e^{\alpha x}(A\sin\beta x+B\cos\beta x)$
        \end{itemize}
        Particular integral: 
        \begin{itemize}[noitemsep, topsep=0pt]
            \item Case $f(x)$ is polynomial: try polynomial with same degree (or higher if needed)
            \item Case $f(x)=Ce^{kx}$: try $y_p=De^{kx}$ (or $Dxe^{kx}$, or $Dx^2e^{kx}$)
            \item Case $f(x)=C_1\sin{kx}+C_2\cos{kx}$: try $y_p=D_1\sin{kx}+D_2\cos{kx}$ (or $D_1x\sin{kx}+D_2x\cos{kx}$)
        \end{itemize}
    \item Partial differentiation:
        \begin{itemize}[topsep=0pt]
            \item Differential relations:\smallbreak
                $df=\left(\dfrac{\partial f}{\partial x}\right)_ydx+\left(\dfrac{\partial f}{\partial y}\right)_xdy$
            \item Chain rule:\smallbreak
                $\left(\dfrac{\partial f}{\partial u}\right)_v=\left(\dfrac{\partial f}{\partial x}\right)_y\left(\dfrac{\partial x}{\partial u}\right)_v+\left(\dfrac{\partial f}{\partial y}\right)_x\left(\dfrac{\partial y}{\partial u}\right)_v$\smallbreak
                $\left(\dfrac{\partial f}{\partial v}\right)_u=\left(\dfrac{\partial f}{\partial x}\right)_y\left(\dfrac{\partial x}{\partial v}\right)_u+\left(\dfrac{\partial f}{\partial y}\right)_x\left(\dfrac{\partial y}{\partial v}\right)_u$
            \item Reciprocity:\smallbreak
                $\left(\dfrac{\partial x}{\partial y}\right)_z\left(\dfrac{\partial y}{\partial x}\right)_z=1$
            \item Cyclic relations:\smallbreak
                $\left(\dfrac{\partial x}{\partial z}\right)_y\left(\dfrac{\partial y}{\partial x}\right)_z\left(\dfrac{\partial z}{\partial y}\right)_x=-1$
            \item Exact differential:\\
                $P(x,y)dx+Q(x,y)dy$ is an exact differential iff $\dfrac{\partial P}{\partial y}=\dfrac{\partial Q}{\partial x}$.
            \item Taylor series:\smallbreak
                $\begin{aligned}
                    f(x,y) &= f(x_0,y_0)\\
                           &+ f_x(x_0,y_0)(x-x_0)+f_y(x_0,y_0)(y-y_0)\\
                           &+ \dfrac{1}{2!}\big(f_{xx}(x_0,y_0)(x-x_0)^2+2f_{xy}(x_0,y_0)(x-x_0)(y-y_0)+f_{yy}(x_0,y_0)(y-y_0)^2\big)+\cdots
                \end{aligned}$
        \end{itemize}
    \item Stationary points of multi-variable functions: $f$ has a stationary point if $f_x=f_y=0$
        \begin{itemize}[noitemsep, topsep=0pt]
            \item Local minimum: $f_{xx}f_{yy}>f_{xy}^2$ with $f_{xx}>0$ and $f_{yy}>0$
            \item Local maximum: $f_{xx}f_{yy}>f_{xy}^2$ with $f_{xx}<0$ and $f_{yy}<0$
            \item Saddle point: $f_{xx}f_{yy}<f_{xy}^2$
        \end{itemize}
    \item Lagrange multipliers and Lagrangian function:\\
        To find the stationary points of $f(x,y)$ subject to the constraint $g(x,y)=0$:\\
        define the Lagrangian function
            \[L(x,y,\lambda)=f(x,y)-\lambda g(x,y)\]
        and solve $L_x=L_y=L_\lambda=0$.
    \item Gradient of a scalar field:\\
        $\bm{\nabla}\Phi=(\Phi_x,\Phi_y,\Phi_z)$\\
        $d\Phi=(\bm{\nabla}\Phi)\cdot d\mathbf{x}$
    \item Divergence of a vector field:\smallbreak
        div $\mathbf{F}=\bm{\nabla}\cdot\mathbf{F}=\left(\dfrac{\partial}{\partial x},\dfrac{\partial}{\partial y},\dfrac{\partial}{\partial z}\right)\cdot(F_x,F_y,F_z)=\dfrac{\partial F_x}{\partial x}+\dfrac{\partial F_y}{\partial y}+\dfrac{\partial F_z}{\partial z}$\smallbreak
        $\bm{\nabla}\cdot(\bm{\nabla}\Phi)=\nabla^2\Phi=\dfrac{\partial^2\Phi}{\partial x^2}+\dfrac{\partial^2\Phi}{\partial y^2}+\dfrac{\partial^2\Phi}{\partial z^2}$
    \item Curl of a vector field:\smallbreak
        curl $\mathbf{F}=\bm{\nabla}\times\mathbf{F}=\left(\dfrac{\partial}{\partial x},\dfrac{\partial}{\partial y},\dfrac{\partial}{\partial z}\right)\times(F_x,F_y,F_z)=\left(\dfrac{\partial F_z}{\partial y}-\dfrac{\partial F_y}{\partial z},\dfrac{\partial F_x}{\partial z}-\dfrac{\partial F_z}{\partial x},\dfrac{\partial F_y}{\partial x}-\dfrac{\partial F_x}{\partial y}\right)$\\
        $\bm{\nabla}\times(\bm{\nabla}\Phi)=\mathbf{0}$
    \item Normal to a surface:\smallbreak
        $\mathbf{n}=\dfrac{\bm{\nabla}\Phi}{|\bm{\nabla}\Phi|}$
    \item Line integral of a scalar field:\smallbreak
        $\displaystyle\int_\Gamma\Phi ds=\displaystyle\int_{s_1}^{s_2}\Phi(\mathbf{x}(s))ds=\displaystyle\int_{t_1}^{t_2}\Phi(\mathbf{x}(t))\left|\dfrac{d\mathbf{x}}{dt}\right|dt$
    \item Line integral of a vector field:\smallbreak
        $\displaystyle\int_\Gamma\mathbf{F}(\mathbf{x})\cdot d\mathbf{x}=\displaystyle\int_{t_1}^{t_2}\mathbf{F}(\mathbf{x}(t))\cdot\dfrac{d\mathbf{x}}{dt}dt$
    \item Conservative vector fields: $\mathbf{F}=\bm{\nabla}\Phi$\smallbreak
        $\displaystyle\int_\Gamma\mathbf{F}\cdot d\mathbf{x}=\displaystyle\int_\Gamma(\bm{\nabla}\Phi)\cdot d\mathbf{x}=\Phi(\mathbf{x}_1)-\Phi(\mathbf{x}_2)$\smallbreak
        $\displaystyle\oint_\Gamma\mathbf{F}\cdot d\mathbf{x}=\displaystyle\oint_\Gamma(\bm{\nabla}\Phi)\cdot d\mathbf{x}=0$
    \item Surface integral (flux):\smallbreak
        $\displaystyle\int_S\mathbf{F}\cdot d\mathbf{S}=\displaystyle\int_S\mathbf{F}\cdot\mathbf{n}dS$
    \item Gauss's theorem (divergence theorem):\smallbreak
        $\displaystyle\int_V(\bm{\nabla}\cdot\mathbf{F})dV=\displaystyle\int_S\mathbf{F}\cdot d\mathbf{S}$, where $S$ is the bounding surface of $V$
    \item Stoke's theorem (curl theorem):\smallbreak
        $\displaystyle\int_S(\bm{\nabla}\times\mathbf{F})\cdot d\mathbf{S}=\displaystyle\int_C\mathbf{F}\cdot d\mathbf{x}$, where $C$ is the boundary of $S$ (also called $\partial S$)
    \newpage
    \item Decomposing matrix $\mathbf{M}$ as sum of a symmetric matrix $\mathbf{S}$ and an anti-symmetric matrix $\mathbf{A}$: \\
        $\mathbf{S}=\frac{1}{2}(\mathbf{M}+\mathbf{M}^\top)$, $\mathbf{A}=\frac{1}{2}(\mathbf{M}-\mathbf{M}^\top)$\\
        For an anti-symmetric matrix $\mathbf{A}$, $\mathbf{x}^\top\mathbf{Ax}=0$ for any column vector $\mathbf{x}$.
    \item Hermitian conjugation:\\
        If $\mathbf{A}=(a_{ij})$, then the Hermitian conjugate is $\mathbf{A}^\dagger=(\mathbf{A}^\top)^*=(\mathbf{A}^*)^\top=(a^*_{ji})$\\
        Hermitian matrix: $\mathbf{A}^\dagger=\mathbf{A}$
    \item Trace: 
        For an $n\times n$ matrix $\mathbf{A}$, trace($\mathbf{A}$)=$\displaystyle\sum_{i=1}^n a_{ii}$\\
        The trace of the product of a symmetric and an antisymmetric matrix is 0.\\
        trace($\mathbf{AB}$)=trace($\mathbf{BA}$).\\
        The results can be generalised and holds for any cyclic permutation of the order of multiplication.
    \item Minors and cofactors:\\
        For an $n\times n$ matrix $\mathbf{A}=(a_{ij})$, let $\mathbf{M}_{ij}$ be an $(n-1)\times(n-1)$ sumathbfatrix:
        \begin{itemize}[noitemsep, topsep=0pt]
            \item Minor of the element $a_{ij}$ of $\mathbf{A}$: $|\mathbf{M}_{ij}|$
            \item Cofactor of $a_{ij}$: $A_{ij}=(-1)^{i+j}|\mathbf{M}_{ij}|$
                \[\begin{pmatrix}
                    + & - & + & \dots \\
                    - & + & - & \dots \\
                    + & - & + & \dots \\
                    \vdots & \vdots & \vdots & \ddots
                \end{pmatrix}\]
            \item Classical adjoint: (adj$\mathbf{A}$)$_{ij}=A_{ji}$
                \[\begin{pmatrix}
                    A_{11} & A_{21} & \dots  & A_{j1} & \dots  & A_{n1}\\
                    A_{12} & A_{22} & \dots  & A_{j2} & \dots  & A_{n2}\\
                    \vdots & \vdots & \ddots & \vdots & \ddots & \vdots\\
                    A_{1i} & A_{2i} & \dots  & A_{ji} & \dots  & A_{ni}\\
                    \vdots & \vdots & \ddots & \vdots & \ddots & \vdots\\
                    A_{1n} & A_{2n} & \dots  & A_{jn} & \dots  & A_{nn}\\
                \end{pmatrix}\]
        \end{itemize}
    \item Determinants:\smallbreak
        $|\mathbf{A}|=\displaystyle\sum_{j=1}^n a_{ij}A_{ij}$ for any fixed $i$; or\\
        $|\mathbf{A}|=\displaystyle\sum_{i=1}^n a_{ij}A_{ij}$ for any fixed $j$; or\smallbreak
        product of the elements on the diagonal if the matrix is triangular.
        \begin{itemize}[topsep=0pt]
            \item $\mathbf{A}(\text{adj}\mathbf{A})=(\text{det}\mathbf{A})\mathbf{I}$
            \item Interchanging any two rows or columns of $\mathbf{A}$ changes the sign of $\text{det}\mathbf{A}$
            \item $\text{det}\mathbf{A}=0$ if any two rows or columns are the same
            \item Multiplying all the elements of any one row or column of $\mathbf{A}$ by $\lambda$ multiplies $\text{det}\mathbf{A}$ by $\lambda$
            \item Adding a multiple of row (column) on another row (column) leaves $\text{det}\mathbf{A}$ unchanged
            \item $\text{det}\mathbf{AB}=(\text{det}\mathbf{A})(\text{det}\mathbf{B})$
            \item $\text{det}\mathbf{A}=\text{det}\mathbf{A}^\top$
        \end{itemize}
    \item Eigenvalues:
        \begin{itemize}[topsep=0pt]
            \item $\text{det}\mathbf{A}=\displaystyle\prod_{i=1}^n \lambda_i$
            \item trace$(\mathbf{A})=\displaystyle\sum_{i=1}^n \lambda_i$
        \end{itemize}
    \item Diagonalisation of real symmetric matrices: if $\mathbf{A}$ is real symmetric, then:
        \begin{itemize}[topsep=0pt]
            \item $\mathbf{A}$ has $n$ distinct eigenvalues $\lambda_1$, $\lambda_2$, ..., $\lambda_n$;
            \item $\mathbf{A}$ has $n$ linearly independent eigenvectors $\mathbf{e}_1$, $\mathbf{e}_2$, ..., $\mathbf{e}_n$ that form an orthonormal basis;
            \item $\mathbf{A}$ can be diagonalised by setting $\mathbf{X}=\begin{pmatrix}
                    \mathbf{e}_1 & \mathbf{e}_2 & \dots & \mathbf{e}_n
                \end{pmatrix}$, then \smallbreak
                $\mathbf{A}'=\mathbf{X}^\top\mathbf{AX}=
                \begin{pmatrix}
                    \lambda_1 & 0 & \dots & 0\\
                    0 & \lambda_2 & \dots & 0\\
                    \vdots & \vdots & \ddots & \vdots\\
                    0 & 0 & \dots & \lambda_n 
                \end{pmatrix}$
        \end{itemize}
\end{enumerate}
\end{document}
